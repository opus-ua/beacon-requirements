\section{Requirements}
    \subsection{Functional Requirements - Frontend}
        \subsubsection{View Map}
            \subsubsubsection{Inputs}
                \textit{Geotag:} The current location of the user.

            \subsubsubsection{Processing}
                Poll the backend for local beacons.

            \subsubsubsection{Outputs}
                The frontend will display a map overlaid with the beacons returned
                by the backend.

            \subsubsubsection{Error Handling}
                If a geotag cannot be acquired, the frontend will display an error
                message. \newline
                If the frontend cannot establish a connection with the backend,
an error message will be displayed. \newline
If the backend returns a specific error message, a user-friendly
version of the message will be displayed.


        \subsubsection{Post Beacon}
            \subsubsubsection{Inputs}
                \textit{Image:} A picture taken in-app by the user. \newline
                \textit{Description:} An optional string describing the image. \newline
                \textit{Geotag:} The current location of the user.

            \subsubsubsection{Processing}
                The frontend will submit the provided information to the backend.

            \subsubsubsection{Outputs}
                The frontend will display the thread view of the newly created beacon.

            \subsubsubsection{Error Handling}
                If a geotag cannot be acquired, the frontend will display an error
                message. \newline
                If the frontend cannot establish a connection with the backend,
an error message will be displayed. \newline
If the backend returns a specific error message, a user-friendly
version of the message will be displayed.


        \subsubsection{View Comment Thread}
            \subsubsubsection{Inputs}
                \textit{Beacon:} The beacon whose comment thread is to be displayed.

            \subsubsubsection{Processing}
                The full comment thread and full-size image will be requested from
                the backend.

            \subsubsubsection{Outputs}
                The comment thread for the beacon in question will be displayed.

            \subsubsubsection{Error Handling}
                If the frontend cannot establish a connection with the backend,
an error message will be displayed. \newline
If the backend returns a specific error message, a user-friendly
version of the message will be displayed.


        \subsubsection{Heart Post}
            \subsubsubsection{Inputs}
                \textit{Post:} The post to be hearted.

            \subsubsubsection{Processing}
                A request is sent to the backend to heart the post in question.

            \subsubsubsection{Outputs}
                The heart icon is visually filled.

            \subsubsubsection{Error Handling}
                If the frontend cannot establish a connection with the backend,
an error message will be displayed. \newline
If the backend returns a specific error message, a user-friendly
version of the message will be displayed.


        \subsubsection{Post Comment}
            \subsubsubsection{Inputs}
                \textit{Beacon:} The beacon to which the comment will be appended.
                                \newline
                \textit{Comment:} The textual body of the comment to be posted.

            \subsubsubsection{Processing}
                The comment is submitted to the backend.

            \subsubsubsection{Outputs}
                The thread is updated to include the newly posted comment.

            \subsubsubsection{Error Handling}
                If the frontend cannot establish a connection with the backend,
an error message will be displayed. \newline
If the backend returns a specific error message, a user-friendly
version of the message will be displayed.


        \subsubsection{View User Profile}
            \subsubsubsection{Inputs}
                \textit{User ID:} The ID of the user whose profile is to be
                                        displayed.

            \subsubsubsection{Processing}
                The profile information of the user in question will be requested from
                the backend.

            \subsubsubsection{Outputs}
                The profile of the requested user will be displayed.

            \subsubsubsection{Error Handling}
                If the frontend cannot establish a connection with the backend,
an error message will be displayed. \newline
If the backend returns a specific error message, a user-friendly
version of the message will be displayed.


        \subsubsection{Flag Post}
            \subsubsubsection{Inputs}
                \textit{Post:} The post to be flagged. \newline
                \textit{Reason:} The reason that this post is being flagged.

            \subsubsubsection{Processing}
                The flagging is submitted to the backend. 

            \subsubsubsection{Outputs}
                A message indicating that the flag has been submitted is displayed
                to the user.

            \subsubsubsection{Error Handling}
                If the frontend cannot establish a connection with the backend,
an error message will be displayed. \newline
If the backend returns a specific error message, a user-friendly
version of the message will be displayed.


        \subsubsection{Account Creation}
            \subsubsubsection{Inputs}
                \textit{Credentials:} Authorization information generated by a third
                                        party service. \newline
                \textit{Username:} The desired username of the new user.

            \subsubsubsection{Processing}
                A request to create a new account is submitted to the backend.

            \subsubsubsection{Outputs}
                Upon success, map view is displayed.

            \subsubsubsection{Error Handling}
                If an existing user already has the requested username, the new user
                will be requested to input another username and the submission process
                is repeated. \newline
                If the frontend cannot establish a connection with the backend,
an error message will be displayed. \newline
If the backend returns a specific error message, a user-friendly
version of the message will be displayed.


    \subsection{Use Cases}
        \begin{figure}[H]
            \centering
            \includegraphics{tmp/frontend-use-case.1} 
            \caption{Frontend Use Case Diagram}
        \end{figure}

        \begin{figure}[H]
            \centering
            \includegraphics{tmp/backend-use-case.1} 
            \caption{Backend Use Case Diagram}
        \end{figure}

        Note that due to the tight integration between the frontend and backend,
        the two use case diagrams above mirror each other quite closely.
        Nearly every action possible within the frontend requires data from the
        backend and every function of the backend is intended to be used by the
        frontend.

        \subsubsection{View Map}
        \begin{figure}[H]
            \centering
            \includegraphics{tmp/activity-view-map.1} 
            \caption{Activity Diagram for the View Map use case}
        \end{figure}

        \paragraph*{}
        When the user opens map view, the frontend will take a geotag of the device's
        current location. This geotag is used to construct a REST request
        for the beacons near these coordinates.
        The backend receives this REST request and polls the database for the desired
        beacons. The resulting list is formatted into a REST response and sent back
        to the frontend. The frontend then displays this list of beacons as icons
        overlaid on a map.

        \subsubsection{Post Beacon}
        \subsubsection{View Comment Thread}
        \subsubsection{Heart Post}
        \subsubsection{Post Comment}
        \subsubsection{View User Profile}
        \subsubsection{Follow User}
        \subsubsection{Flag Post}
        \subsubsection{Create Account}


    \subsection{Class Diagrams}
        \subsubsection{Frontend Class Diagram}
            \begin{figure}[H]
                \centering
                \includegraphics{tmp/classes.1} 
                \caption{Frontend Class Diagram}
            \end{figure}

        \subsubsection{Backend Class Diagram}

    \subsection{Non-functional Requirements}

